%At the precipice of computation sits a golden fleece of the modern era. Quantum Computation is a buzz word in academia, and industry alike. Those who misunderstand this theoretical machine 
%herald it as an end-all that will solve all the modern problems in complexity theory. Those who understand it's power are able to approach it rationally, and determine which problems it 
%can solve with increased efficiency compared with it's modern Turing alternative. At the end of the day there are some problems for which the doped silicon of electronic computers simply outperforms the polarized glass of their quantum companions. it is up to those with a theoretical understanding of the mathematical, and physical properties of the proposed machine to determine where beneficiary algorithms might overlap and allow for substantial speed up using this new paradigm of Quantum Computation.

There has been a buzz in the world of computer science lately about
quantum computing \cite{sciam}. Since being heralded as the most
efficient way to simulate a quantum physics system, research into
algorithms for quantum computers has investigated topics ranging from
pattern search to the Fourier transform.  The speeedup attained on
quantum computers by this myriad of algorithms demonstrates their
promise as a next-generation computing resource. 


The realm of numerical analysis has so-far been slow to adopt quantum
computing, even though numerical algorithms in science and engineering
routinely push the limits of traditional computation speed.  As a
result, numerical algorithms have experienced a dramatic expansion
into the use of graphics processors and other ``exotic'' computing
hardware to meet the ever-increasing computational demands.
However even with these new hardware, numerical methods continue to
rely performance improvements due to technological advances that
reduce the size of transistors.  Unfortunately this advance has a
rapidly-approaching limit, the size of an atom.  Quantum computation,
on the other hand, utilizes an entirely different computing paradigm,
and might just provide an answer to this atomic size limitation.

This transition is not seamless, however, as there are a number of
differences between traditional electronic, and quantum, computers.  
One of which is known as the \emph{no cloning property}, and holds
that no qubit (Quantum Binary Digit) may be cloned, this limits fanout
(which is the branching off of a wire in a computer).  Another
challenge of quantum computers is that no memory elements can exist. 
This makes the implementation of numerical methods more, as they
typically rely heavily on memory.


