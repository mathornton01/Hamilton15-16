%At the precipice of computation sits a golden fleece of the modern era. Quantum Computation is a buzz word in academia, and industry alike. Those who misunderstand this theoretical machine 
%herald it as an end-all that will solve all the modern problems in complexity theory. Those who understand it's power are able to approach it rationally, and determine which problems it 
%can solve with increased efficiency compared with it's modern Turing alternative. At the end of the day there are some problems for which the doped silicon of electronic computers simply outperforms the polarized glass of their quantum companions. it is up to those with a theoretical understanding of the mathematical, and physical properties of the proposed machine to determine where beneficiary algorithms might overlap and allow for substantial speed up using this new paradigm of Quantum Computation.

There has been a buzz in the world of computer science lately about quantum computing. \cite{sciam}
Since they were heralded as the most efficient way to simulate a quantum physics system, quantum computers have had algorithms developed for everything from searching to performing the Fourier transform. 
This diversity in algorithms that undergo a speedup when using a quantum computer demonstrates it's usefulness. 


In numerical analysis we have come up on some walls in terms of traditional computation speed. 
There is a big push towards the use of graphics processors as more powerful processors. 
As transistors continue to decrease in size, these algorithms perform better and better. 
But there is a limit, the size of an atom. 
Quantum computation might be just the answer that we are looking for. 


There are however a few minor concerns when using a quantum computer. 
One of which is known as the no cloning property, and holds that no qubit (Quantum Binary Digit) may be cloned, this limits fanout (which is the branching off of a wire in a computer). 
Another property is that no memory elements can exist. 
This makes the implementation of numerical methods very tedious, as a majority of them rely heavily on memory.


